\begin{abstract}
    Just-in-Time compilation has allowed for significant performance gains during the run-time of applications.
    LLVM can be embedded within an application to allow JIT compilation during run-time.
    Similarly, the OMR JIT compiler can be embedded within an application.
    Both frameworks offer simple, programmer interaces to define and generate native methods at run-time.
    In this report we discuss the different approaches the two frameworks employ.
    We then measure the overhead associated with each framework while compiling relatively simple functions.
    \textcolor{blue}{
    We found that while LLVM required a significantly larger memory footprint, it was able to generate code more quickly.
    Furthermore, the code LLVM generated was by default more optimized.
    By configuring JITBuilder, we were able to reduce the compilation time significantly, however it still was above that of LLVM.
    }
\end{abstract}
  
