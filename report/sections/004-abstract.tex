\begin{abstract}
    Just-in-Time compilation has allowed for significant performance gains during the run-time of applications.
    LLVM can be embedded within an application to allow JIT compilation during run-time.
    Similarly, the OMR JIT compiler can be embedded within an application.
    Both frameworks offer simple, programmer interaces to define and generate native methods at run-time.
    In this report we discuss the different approaches the two frameworks employ.
    We then measure the overhead associated with each framework while compiling relatively simple functions.
    \textcolor{blue}{
    We found that while LLVM required a larger memory footprint, in certain cases it was able to generate code more quickly.
    Furthermore, in our tests, the code LLVM generated typically offered high throughput.
    JitBuilder, a relatively young project compared to LLVM, does not expose the underlying configuration of TRJIT, instead locking compilations in to the \textit{warm} optimization level.
    }
\end{abstract}
  
