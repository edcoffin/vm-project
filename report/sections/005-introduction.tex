\section{Introduction}
\label{sec:introduction}
Just-in-time compilation, or JIT compilation, is a technique to improve the run-time binary translation of an application \cite{Aycock:2003:BHJ:857076.857077}.
Using information collected from the running application, JIT compilers can further optimize generated code.
For instance, by profiling which code paths are executed most frequently, JIT compilers can generate code that is optimized for hot-paths \cite{Smith:2005:VMV:1204009}.
A JIT compiler might also inline entire blocks of code, generate an inline cache to speed up the dispatch of polymorphic method calls \cite{picPaper}, or generate hardware-specific instructions.

Runtimes for the high-level language Java make heavy use of JIT compilers, allowing their workloads to execute much faster than if they were entirely interpreted \cite{HiPerfJava}.
Given that JIT compilation can add significant overhead to a workload, compilation is typically applied selectively to the code executed most frequently.
Furthermore, given that compilation can be viewed as a continuum, where slow, unoptimized code is cheap to generate, and where fast, optimized code is expensive to generate, a compiler will often support multiple optimization levels \cite{Smith:2005:VMV:1204009}.
A runtime with such an optimizing JIT compiler can then employ a staged compilation strategy which would allow the runtime to apply compilation and optimization according to heuristics \cite{dynamo}.\footnote{
    An example of a staged, or tiered compilation strategy, can be seen with the Testarossa JIT compiler (TRJIT) in the Eclipse OpenJ9 JVM \cite{eclipseOMR,TRJITOptimize}.
    Here an invocation threshold must be met before the JIT compiler will compile a method.
    Additionally, separate thresholds can be associated with each optimization level \cite{TRJitLevels}.
    We will discuss TRJIT in more detail when we look at JitBuilder in \hyperref[sec:jitbuilder]{Section 2.2}.}

Considering JIT compilers may also want to perform some of the optimizations found in a static compiler such as common sub-expression elimination, loop unrolling and constant propagation \cite{compilerBook}, an application developer, or language designer interested in enhancing a runtime's performance by adding a JIT compiler, will have a large engineering task ahead of them.
In addition, considering that JIT compilers often need to generate native, or architecture-specific code, for multiple platforms, the effort required can increase significantly.
It is no surprise then that libraries or frameworks encapsulating proven, high-performance JIT compilers are available to software engineers today.
For an engineer interested in incorporating an existing JIT framework, it is useful to consider the following questions when making their selection: 
\begin{itemize}
    \item How much processor overhead does the JIT compiler have?
    \item What is the quality of the generated JIT code and what effect does this have on throughput?
    \item How is the resident state set (RSS) for the application effected?    
    \item How much larger is the application binary with the inclusion of the JIT framework?
    \item How configurable is the JIT framework?
    \item How easy is it for the programmer to incorporate and use the JIT framework?
\end{itemize} 

In this report, we will consider those questions while looking at two such JIT frameworks: LLVM MCJIT \cite{LLVM_Web} and OMR JitBuilder \cite{jitbuilderPaper}.
In Sections \hyperref[sec:llvm]{2} and \hyperref[sec:jitbuilder]{3}, we will discuss the background of each compiler and consider the techniques they employ.
In \hyperref[sec:methodology]{Section 3}, we outline the methods we used to answer those questions.
In \hyperref[sec:results]{Section 4}, we will consider the results.
In \hyperref[sec:related-work]{Section 5} we will discuss related work.
In \hyperref[sec:future-work]{Section 6}, we will look at potential future work, and finally, in \hyperref[sec:summary]{Section 7}, we will summarize the report.