\subsection{JitBuilder}
JitBuilder is the embeddable framework for interacting with the Eclipse OMR JIT compiler Testarossa \cite{jitbuilderPaper}.
Eclipse OMR is an open-source collection of components for building language runtime environments.
Some of OMR's components include a garbage collection framework, a thread library, cross-platform port support, virtual machine building blocks, and a JIT compiler \cite{eclipseOMR,RebuildingAirliner}.
Much of the infrastructure driving Eclipse OpenJ9, a popular, open-source Java Virtual Machine, links to OMR components.

Through JitBuilder's API, users generate IR upon which optimizations are applied and from which native code is generated \cite{SuganumaIBMJit}.
Based on basic-blocks, the IR is arranged into directed acyclic graphic (DAG) structures called trees, composed of nodes which each contain an opcode.
OMR IR has several hundred opcodes, where typically with each type operation has a code for each data type: integer, pointer, double, float, vector, etc...\footnote{
    This can be contrasted with LLVM's roughly 31 opcodes, where the data type code is instead embedded within the instruction.
}
The children of these opcode nodes are in turn operands.
Trees with side-effects, which cannot be reordered, belong to a list of elements called treetops \cite{treetops}.
Existing nodes can be reused within a given tree, making optimizations such as common subexpression elimination relatively simple.
Though Testarossa supports several levels of optimization ranging from \textit{cold}, with roughly 20 optimizations applied, all the way to \textit{scorching}, where as many as 170 optimizations are applied, as of writing, JitBuilder is locked at the \textit{Warm} optimization level \cite{sanchez2011using, jitbuilderWarm}.



