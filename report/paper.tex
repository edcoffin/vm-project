\documentclass[sigconf]{acmart}

\usepackage{pgfplots}
\usepackage[utf8]{inputenc}
\usepackage{xcolor}
\usepackage{multirow}
\usepackage{hyperref}
\usepackage{listings}
\usepackage{booktabs}
\usepackage[group-separator={,}]{siunitx}

\pgfplotsset{compat=1.8}
\usepgfplotslibrary{statistics}

\lstset{
language=C,
basicstyle=\scriptsize\ttfamily,
commentstyle=\ttfamily\color{gray},
numbers=left,
numberstyle=\ttfamily\color{gray}\footnotesize,
stepnumber=1,
numbersep=5pt,
backgroundcolor=\color{white},
showspaces=false,
showstringspaces=false,
showtabs=false,
frame=single,
tabsize=2,
captionpos=b,
breaklines=true,
breakatwhitespace=false,
title=\lstname,
escapeinside={},
keywordstyle={},
morekeywords={}
}


\setcopyright{none}
\settopmatter{printacmref=false}

%%%% As of March 2017, [siggraph] is no longer used. Please use sigconf (above) for SIGGRAPH conferences.

%%%% Proceedings format for SIGPLAN conferences 
% \documentclass[sigplan, anonymous, review]{acmart}

%%%% Proceedings format for SIGCHI conferences
% \documentclass[sigchi, review]{acmart}

%%%% To use the SIGCHI extended abstract template, please visit
% https://www.overleaf.com/read/zzzfqvkmrfzn

%%
%% \BibTeX command to typeset BibTeX logo in the docs
\AtBeginDocument{%
  \providecommand\BibTeX{{%
    \normalfont B\kern-0.5em{\scshape i\kern-0.25em b}\kern-0.8em\TeX}}}

%% Rights management information.  This information is sent to you
%% when you complete the rights form.  These commands have SAMPLE
%% values in them; it is your responsibility as an author to replace
%% the commands and values with those provided to you when you
%% complete the rights form.

%%\setcopyright{acmcopyright}
%%\copyrightyear{2019}
%%\acmYear{2019}
%%\acmDOI{10.1145/1122445.1122456}

%% These commands are for a PROCEEDINGS abstract or paper.
%%\acmConference[CASCON '19]{29th Annual International Conference on Computer Science and Software Engineering}{November 2019}{Toronto, Ontario, Canada}
%%\acmBooktitle{Woodstock '18: ACM Symposium on Neural Gaze Detection,
%%  June 03--05, 2018, Woodstock, NY}
%%\acmPrice{15.00}
%%\acmISBN{978-1-4503-9999-9/18/06}


%%
%% Submission ID.
%% Use this when submitting an article to a sponsored event. You'll
%% receive a unique submission ID from the organizers
%% of the event, and this ID should be used as the parameter to this command.
%%\acmSubmissionID{123-A56-BU3}

%%
%% The majority of ACM publications use numbered citations and
%% references.  The command \citestyle{authoryear} switches to the
%% "author year" style.
%%
%% If you are preparing content for an event
%% sponsored by ACM SIGGRAPH, you must use the "author year" style of
%% citations and references.
%% Uncommenting
%% the next command will enable that style.
%%\citestyle{acmauthoryear}

%%
%% end of the preamble, start of the body of the document source.
\begin{document}

\settopmatter{printfolios=true}
%%
%% The "title" command has an optional parameter,
%% allowing the author to define a "short title" to be used in page headers.
\title{A Comparison of JIT Compiler Overhead}


\author{Eric Coffin}
\affiliation{
  \institution{Faculty of Computer Science\\University of New Brunswick}
  \city{Fredericton}
  \state{NB}
  \country{Canada}
}
\email{eric.coffin@unb.ca}


%%
%% By default, the full list of authors will be used in the page
%% headers. Often, this list is too long, and will overlap
%% other information printed in the page headers. This command allows
%% the author to define a more concise list
%% of authors' names for this purpose.
%%\renewcommand{\shortauthors}{Trovato and Tobin, et al.}

%%
%% The code below is generated by the tool at http://dl.acm.org/ccs.cfm.
%% Please copy and paste the code instead of the example below.
%%
\begin{CCSXML}
    <ccs2012>
    <concept>
    <concept_id>10011007.10011006.10011041</concept_id>
    <concept_desc>Software and its engineering~Compilers</concept_desc>
    <concept_significance>500</concept_significance>
    </concept>
    </ccs2012>
\end{CCSXML}
    
\ccsdesc[500]{Software and its engineering~Compilers}

%%
%% Keywords. The author(s) should pick words that accurately describe
%% the work being presented. Separate the keywords with commas.
\keywords{compilers, just-in-time, optimization}

\begin{abstract}
    Just-in-Time compilation can provide significant performance gains for applications.
    Two popular open-source JIT frameworks are MCJIT and JitBuilder. 
    Both of these compilers can be embedded within applications, offering interfaces to define and generate native code at run-time.
    LLVM is a collection of modular compiler and toolchain components, while MCJIT is a framework based on these components to provide JIT compilation.
    Similarly, JitBuilder is framework based on the Eclipse OMR compiler and runtime components.
    In this report we discuss the different approaches these two frameworks employ.
    In addition, we attempt to measure the overhead of each framework while compiling and then executing a small handful of test programs.
    We found that while LLVM required a larger memory footprint, in certain cases it was able to generate code more quickly.
    Furthermore, there were cases where the code LLVM generated offered higher throughput.
    JitBuilder did provide smaller disk and memory footprints and generated faster code than MCJIT in some cases.
\end{abstract}
  


%%
%% This command processes the author and affiliation and title
%% information and builds the first part of the formatted document.
\maketitle

\section{Introduction}


% \begin{figure}
%     \centering
%     \includegraphics[width=7cm]{images/throughput.png}
%     \caption{ Application Rampup. The application moves from the startup phase to the throughput phase once enough methods have been compiled to reach optimal throughput \cite{Sogaro:2017:MLJ:3172795.3172812}.}
%     \label{fig:throughput}
%     \Description[Graph showing application throughput as function of time.]{Initially the application begins in the startup phase, during which all methods are interpreted. After a period of time, the most important code paths will have been compiled, allowing the application to enter an optimal period, or throughput phase.}
% \end{figure}





\section{Background}
In this section we will discuss the background of the two JIT compiler frameworks we are interested in: LLVM MCJIT and OMR JitBuilder.
In particular, we will focus on the motivation for each framework, as well as discuss the techniques and features they provide.
\section{LLVM}
\label{sec:llvm}
LLVM, which at one time stood for Low Level Virtual Machine, is a popular set of open-source, modular compiler and toolchain components \cite{lattner2004llvm}.
The compiler framework was originally designed to provide analysis and transformation for an application throughout it's entire lifetime: from initial compilation and linking, through to runtime and even while the application was offline (see Figure \ref{fig:llvmarch}).
\begin{figure*}
    \includegraphics[width=\textwidth]{images/llvm-architecture.png}
    \caption{ The LLVM Compiler Framework Architecture \cite{lattner2004llvm}.}
    \label{fig:llvmarch}
    \Description[]{}
\end{figure*}
To achieve this ambitious goal, the framework utilizes a well defined, human-readable, intermediate representation called LLVM IR.
The IR, which is initially generated by the front end can be packaged with the target architecture binary along with profiling instructions for later runtime compilation (JIT) as well as more aggressive offline optimizations.
Several important characteristics of LLVM IR as as follows:
\begin{itemize}
    \item The IR maintains Static Single Assignment (SSA) form with unlimited virtual registers.
    \item Each register is of one of four primitive types: boolean, integer, floating-point or pointer.
    \item Similar to RISC, memory operations are carried out in registers, and between registers and memory using Load and Store instructions.
    \item The IR is limited to 31 opcodes.
    \item The IR is organized into basic blocks which must be composed into valid control flow graphs, simplifying the work required for various optimizations. 
\end{itemize}

\begin{lstlisting}[float,floatplacement=H,
caption={LLVM IR for a function multiplying x * y and adding z \cite{LLVM_Jit_Tutorial}},
label=lst:llvm_ir]
define i32 @mul_add(i32 %x, i32 %y, i32 %z) {
    entry:
    %tmp = mul i32 %x, %y
    %tmp2 = add i32 %tmp, %z
    ret i32 %tmp2
}\end{lstlisting}

This report will focus on LLVM's JIT component, which can be accessed through the MCJIT API.
The MCJIT framework provides an API thats accepts IR, generates optimized machine code, and provides a function pointer for calling the generated code.
The JIT compiler offers several levels of optimization: none, less, default, and aggressive, which correspond to O0-O3.
It should be noted that the JIT compiler by default does not perform any IR optimizations or transformations.
Instead, a developer must pass the generated IR to a Pass manager with specific optimizations.
This optimized IR can then be passed to the JIT Engine.

\section{JitBuilder}
\label{sec:jitbuilder}
JITBuilder.

\subsection{LLVM}
\label{sec:llvm}
LLVM, which at one time stood for Low Level Virtual Machine, is a popular set of open-source, modular compiler and toolchain components \cite{lattner2004llvm}.
The compiler framework was originally designed to provide analysis and transformation for an application throughout it's entire lifetime: from initial compilation and linking, through to runtime and even while the application was offline (see Figure \ref{fig:llvmarch}).
\begin{figure*}
    \includegraphics[width=\textwidth]{images/llvm-architecture.png}
    \caption{ The LLVM Compiler Framework Architecture \cite{lattner2004llvm}.}
    \label{fig:llvmarch}
    \Description[]{}
\end{figure*}
To achieve this ambitious goal, the framework utilizes a well defined, human-readable, intermediate representation called LLVM IR.
The IR, which is initially generated by the front end can be packaged with the target architecture binary along with profiling instructions for later runtime compilation (JIT) as well as more aggressive offline optimizations.
Several important characteristics of LLVM IR as as follows:
\begin{itemize}
    \item The IR maintains Static Single Assignment (SSA) form with unlimited virtual registers.
    \item Each register is of one of four primitive types: boolean, integer, floating-point or pointer.
    \item Similar to RISC, memory operations are carried out in registers, and between registers and memory using Load and Store instructions.
    \item The IR is limited to 31 opcodes.
    \item The IR is organized into basic blocks which must be composed into valid control flow graphs, simplifying the work required for various optimizations. 
\end{itemize}

\begin{lstlisting}[float,floatplacement=H,
caption={LLVM IR for a function multiplying x * y and adding z \cite{LLVM_Jit_Tutorial}.},
label=lst:llvm_ir]
define i32 @mul_add(i32 %x, i32 %y, i32 %z) {
    entry:
    %tmp = mul i32 %x, %y
    %tmp2 = add i32 %tmp, %z
    ret i32 %tmp2
}\end{lstlisting}

This report will focus on LLVM's JIT component, which can be accessed through the MCJIT API.
The MCJIT framework provides an API thats accepts IR, generates optimized machine code, and provides a function pointer for calling the generated code.
The JIT compiler offers several levels of optimization: none, less, default, and aggressive.
It should be noted that the JIT compiler by default does not perform any IR optimizations or transformations.
Instead, a developer must pass the generated IR to a PassManager with specific optimizations they intend to apply.
These passes can be categorized as analysis passes, or transformation passes \cite{LLVM_Passes}:
\begin{itemize}
    \item Analysis Passes: collect information about IR for use later by transformations, for debugging or for visualization. 
    A few examples are \textit{print-callgraph}, \textit{print-function}, and \textit{iv-users} for printing the users of a particular induction variable. There are roughly 40 such passes available.
    \item Transformation Passes: These typically modify IR. Examples include \textit{adce} for dead code elimination, \textit{instcombine} for combining redundant instructions, or peephole optimizing, and \textit{tailcallelim} for eliminating tail calls. There are roughly 60 such passes available.
\end{itemize}
This optimized IR will then be used by the ExecutionEngine when generating code for the target architecture.
Before a function is executed by the ExecutionEngine, it first checks if the ObjectCache contains a copy.
If the function could not be found, the compiler will generate code and store it in the ObjectCache before execution \cite{LLVM_MCJIT}.
It is worth noting that an newer JIT API, called ORCJIT, is also part of the LLVM project.
ORCJIT, or On-Request-Compilation JIT, is intended to compliment the MCJIT API -- which compiles eagerly, by adding support for lazy, and concurrent compilations \cite{LLVM_ORCJIT}.

\subsection{JitBuilder}
\label{sec:jitbuilder}
JitBuilder is the embeddable framework for interacting with the Eclipse OMR JIT compiler Testarossa (TRJIT) \cite{jitbuilderPaper}.
Eclipse OMR is an open-source collection of components for building language runtime environments.
Some of OMR's components include a garbage collection framework, a thread library, cross-platform port support, virtual machine building blocks, and the TRJIT compiler \cite{eclipseOMR,RebuildingAirliner}.
Much of the infrastructure driving Eclipse OpenJ9, a popular, open-source Java Virtual Machine, link to OMR components.

Through JitBuilder's API, users generate IR upon which optimizations are applied and from which native code is generated \cite{SuganumaIBMJit}.
Based on basic-blocks, the IR is arranged into directed acyclic graphic (DAG) structures called trees, composed of nodes that each contain an opcode.
The children of these opcode nodes are in turn operands.
While the number of available opcodes is high -- there are several hundred, most of the general instructions (add, load, compare) have a specific opcode for each data type: integer, pointer, double, float, vector, etc...\footnote{
    This can be contrasted with LLVM's roughly 31 opcodes, where the data type code is instead embedded within the instruction.
}
Trees with side-effects, which cannot be reordered, belong to a list of elements called treetops \cite{treetops}.
Existing nodes can be reused within a given tree, making optimizations such as common subexpression elimination relatively simple (see Listing \ref{lst:jitbuilder_ir}).
Though Testarossa supports several levels of optimization ranging from \textit{cold}, with roughly 20 optimizations applied, all the way to \textit{scorching}, whereas as many as 170 optimizations are applied, as of writing, JitBuilder is locked at the \textit{Warm} optimization level \cite{sanchez2011using, jitbuilderWarm}.
\begin{lstlisting}[float,floatplacement=H,
caption={OMR IR representation for (a+b)*(a-b). Note that the iload nodes are reused \cite{treetops}.},
label=lst:jitbuilder_ir]
treetop--> istore a
            |
            imul--- isub-------+
            |                  |
            iadd               |
            |                  |
            +-----> iload b <--+
            |                  |
            +-----> iload a <--+
}\end{lstlisting}



\section{Questions and Methods}
\label{sec:methodology}
To answer the questions we outlined in the introduction, we wrote three simple programs for both LLVM MCJIT and JitBuilder.
For the sake of comparison, we also wrote the programs using standard C++ without any JIT framework.
The programs are as follows:
\begin{itemize}
    \item increment - Calls a single function that adds one to an integer argument.
    \item recursive-fib - Calls a recursive Fibonacci function - \textit{fib(20)}. 
    \item iterative-fib - Calls an iterative Fibonacci function - \textit{fib(20}.  
\end{itemize}
For the JIT implementations, the IR was built using the provided APIs, then native code was generated and later used for the actual function calls.
We built and benchmarked the programs on an x86-64 Linux workstation running Ubuntu 19.04 with 32GB of RAM and an Intel i7-8700 (6-core, 12 thread) processor. 
Both LLVM and OMR were built from source code from their respective GitHub repositories \cite{llvmCommit, omrCommit}.
All the source code for this project can be found online\cite{projectGithub}.

\section{Results}
\label{sec:results}
\subsection{Compilation Time}
First we measured the time to compile each function (see Table \ref{tab:compile-time}).
\begin{table}[]
    \begin{tabular}{@{}lllll@{}}
    \toprule
    Framework  & Program & Mean    & Median  & Std. Dev \\ \midrule
    LLVM       & inc     & 1,228,939 & 1,223,583 & 13,780    \\
    JitBuilder & inc     &   531,730  &  531,976  & 5,292     \\
    LLVM       & rfib    & 1,548,316 & 1,547,491 & 3,925     \\
    JitBuilder & rfib    & 1,855,381 & 1,847,804 & 17,716    \\
    LLVM       & ifib    & 1,711,828 & 1,711,296 & 3,086     \\
    JitBuilder & ifib    & 4,188,082 & 4,183,684 & 11,133    \\ \bottomrule
    \end{tabular}
    \caption{Time to build IR and generate native code.}
    \label{tab:compile-time}
    \end{table}

\section{Related Work}
\label{sec:related-work}
- LLVM IR -> JitBuilder







\section{Future Work}
\label{sec:future-work}
To provide a deeper glimpse into the quality of the generated code, the number of test programs could be expanded to cover such scenarios as arrays, large vector operations, and very large functions.
It would be interesting to compare the performance of repeated JIT compilations rather than tearing down the JIT infrastructure between calls.
The JIT compiler in Eclipse OMR, TRJIT, supports many levels of compilation, as well as support for synchronous, and asynchronous compilation schemes, however, these options have not yet been exposed to JitBuilder.
While we believe JitBuilder should be expanded upon to provide such controls, special attention would have to be given to maintaining the simplicity and usability of the API.
Finally, testing the compilers under more complex scenarios such as making JIT-to-JIT function calls and dealing with code cache limitations would provide more real world results.

\section{Summary}
\label{sec:summary}
Considering the JIT compiler behind JitBuilder was designed specifically for dynamic runtime compilation, it makes sense that it has a lighter footprint than MCJIT, which shares it's codebase with the static compiler for LLVM.
Unlike LLVM's IR, which is meant for life-long usage, the design of OMR's IR does not necessarily have ahead of time, or offline goals in mind.
We believe this simplifies the work required when IR interactions are made in JitBuilder and may give it a performance edge as a dynamic compiler over MCJIT.
The code generated by MCJIT did outperform JitBuilder is some cases.

\bibliographystyle{unsrt}
\bibliography{refs}{}

\end{document}
\endinput
%%
%% End of file `paper.tex'.
